\documentclass[a4paper,12pt]{article}
\usepackage[utf8]{inputenc}
\usepackage[T2A]{fontenc}
\usepackage[russian]{babel}
\usepackage{geometry}
\usepackage{hyperref}
\usepackage{longtable}
\usepackage{caption}
\geometry{margin=2cm}
\usepackage{xcolor}  
\usepackage{listings}

% Настройка стиля для Python
\lstdefinelanguage{Python}{
    keywords={def, return, if, else, for, while, import, as, from, class},
    sensitive=true,
    comment=[l]{\#},
    morestring=[b]",
}
\lstset{
    language=Python,
    basicstyle=\ttfamily\small,
    keywordstyle=\color{blue},
    commentstyle=\color{gray},
    stringstyle=\color{green!60!black},
    showstringspaces=false,
    frame=single,
    numbers=left,
    numberstyle=\tiny,
    numbersep=5pt,
    breaklines=true,
    backgroundcolor=\color{gray!5},
}

\begin{document}

\section*{Отчёт по лабораторной работе 3}

\bigskip

\textbf{Студент:} Кочкожаров Иван Вячеславович

\noindent
\textbf{Группа:} \underline{М8О-308Б-22}

\bigskip

\section{Цель работы}
Исследовать процент совпадения букв при сравнении различных типов текстов: осмысленных и случайных, букв и слов.

\section{Алгоритм сравнения}
\begin{enumerate}
\item Получить две строки одинаковой длины: обрезать или сгенерировать как требуется.
\item Для каждой позиции $i$ от 1 до $N$ ($N$~--- длина): сравнить $t_1[i]$ и $t_2[i]$.
\item Подсчитать число совпадений $C = \sum_{i=1}^N [t_1[i] = t_2[i]]$.
\item Вычислить долю совпадений $P = C / N$.
\end{enumerate}

\section{Описание случаев}
\begin{enumerate}
\item \textbf{Два осмысленных текста}: Платон «Государство» и Мелвилл «Моби Дик». \
\item \textbf{Осмысленный текст и случайные буквы}: Платон и случайная строка букв. \
\item \textbf{Осмысленный текст и случайные слова}: Платон и строка из случайных слов. \
\item \textbf{Два текста из случайных букв}. \
\item \textbf{Два текста из случайных слов}.
\end{enumerate}

\section{Результаты эксперимента}
\captionsetup[longtable]{labelformat=empty}
\begin{longtable}{|p{3cm}|p{3cm}|p{3cm}|p{3cm}|}
\hline
Случай & Длина текста & Совпадения & Доля $P$ \\
\hline
\endhead

1. Осмысленные тексты & 1 238 632 & 80 113 & 0,06471 \\
\hline
2. Осмысленный и случайные буквы & 1 238 632 & 18 355 & 0,01482 \\
\hline
3. Осмысленный и случайные слова & 1 238 632 & 71 085 & 0,05739 \\
\hline
4. Два случайных буквенных текста & 1 000 000 & 19 230 & 0,01923 \\
\hline
5. Два случайных текстов из слов & 1 000 000 & 57 870 & 0,05787 \\
\hline
\end{longtable}

\section{Обсуждение результатов}
На основе полученных значений можно сделать следующие выводы:

\begin{itemize}
\item \textbf{Два осмысленных текста:} значение $P\approx0{,}065$ существенно выше, чем для случайных наборов символов. Это объясняется схожестью статистического распределения букв и пробелов в естественных текстах одного языка: часто встречаются одни и те же буквы, знаки препинания и пробелы.

\item \textbf{Осмысленный текст и случайные буквы:} $P\approx0{,}015$ близко к $1/52\approx0{,}0192$, что соответствует совпадению одного конкретного символа при равновероятном выборе из 52 букв латинского алфавита. Это значит, что вероятность угадать букву случайно примерно такая же.

\item \textbf{Два текста из случайных букв:} $P\approx0{,}0192$ совпадает с теоретическим значением $\sum_{a}p(a)^2$ для равномерного распределения букв (каждая буква встречается с вероятностью $1/52$). Таким образом, эксперимент подтверждает теорию: вероятность совпадения двух случайных букв равна $1/52$.

\item \textbf{Осмысленный текст и случайные слова:} $P\approx0{,}0574$. Здесь сравнивается осмысленный текст с текстом, составленным из случайных английских слов.
\begin{itemize}
  \item Пробелы между словами занимают существенную долю позиций (около 0.15 -- 0.2), что сама по себе даёт высокий шанс совпадения пробела.
  \item Даже если слова в словаре распределены равномерно, повторяются конструкции вроде ``ing'', ``ion'' в различных словах, что даёт совпадения на уровне буквосочетаний.
\end{itemize}

\item \textbf{Два текста из случайных слов:} $P\approx0{,}0579$. При сравнении двух независимых последовательностей случайных слов:
\begin{itemize}
  \item Пробелы в обоих текстах совпадают с одинаковой частотой, добавляя вклад в общий процент.
  \item Несмотря на большой объём словаря, повторяющиеся окончания (например, суффиксы ``ing'', ``ly'') могут совпасть между двумя текстами.
  \item В итоге комбинация совпадающих пробелов и повторяющихся морфем даёт долю совпадений около 0.058, получается очень похожая на предыдущий случай картина. 
\end{itemize}


\end{itemize}

\section{Выбор длины текста}
Для устойчивой оценки доли необходимо, чтобы дисперсия доли совпадений была мала: $\mathrm{Var}(P)=p(1-p)/N$. 
Желаема точность: $\sigma_P = \sqrt{\frac{p(1-p)}{N}} \le \varepsilon = 0.001$. Для желаемой точности $\pm0{,}001$ при $p=0.065$ (наибольшее значение в экспериментах) достаточно $N\approx p(1-p)/(0{,}001)^2\approx0{,}065\cdot0{,}935/10^{-6}\approx6\times10^4$. Поэтому длины порядка $10^5$ символов достаточно для всех сценариев.

\section{Приложение: код}


\begin{lstlisting}
import random
import string
import urllib.request


CNT_RANDOM_TEXTS = 10
LEN_RANDOM_TEXT = 10 ** 6
URL1 = 'https://www.gutenberg.org/cache/epub/1497/pg1497.txt'
URL2 = 'https://www.gutenberg.org/cache/epub/2701/pg2701.txt'

def match_stat(text1, text2):
    cnt = 0
    for char1, char2 in zip(text1, text2):
        if char1 == char2:
            cnt += 1
    return cnt / min(len(text1), len(text2))


def gen_random_letters(n):
    return ''.join(random.choice(string.ascii_letters) for _ in range(n))

def gen_random_words(n):
    url = 'https://raw.githubusercontent.com/dwyl/english-words/master/words.txt'
    response = urllib.request.urlopen(url)
    words = response.read().decode()
    words = words.splitlines()
    text = ''
    while len(text) < n:
        text += ' ' + random.choice(words)
    rem = len(text) - n
    if rem != 0:
        text = text[:-rem]
    return text

def case1():
    print("Case #1: two meaningful texts in natural language.")
    response = urllib.request.urlopen(URL1)
    text1 = response.read().decode()
    print(f"Text 1 len: {len(text1)}")
    response = urllib.request.urlopen(URL2)
    text2 = response.read().decode()
    print(f"Text 2 len: {len(text2)}")
    min_len = min(len(text1), len(text2))
    text1 = text1[:min_len]
    text2 = text2[:min_len]
    print("Text length: {0}".format(min_len))
    print("Match: {0}".format(match_stat(text1, text2)))


def case2():
    print("Case #2: meaningful text and text from random letters.")
    response = urllib.request.urlopen(URL1)
    text1 = response.read().decode()
    s = 0
    for i in range(CNT_RANDOM_TEXTS):
        text2 = gen_random_letters(len(text1))
        s += match_stat(text1, text2)
    s /= CNT_RANDOM_TEXTS
    print("Text length: {0}".format(len(text1)))
    print("Match: {0}".format(s))


def case3():
    print("Case #3: meaningful text and text from random words.")
    response = urllib.request.urlopen(URL1)
    text1 = response.read().decode()
    s = 0
    for i in range(CNT_RANDOM_TEXTS):
        text2 = gen_random_words(len(text1))
        s += match_stat(text1, text2)
    s /= CNT_RANDOM_TEXTS
    print("Text length: {0}".format(len(text1)))
    print("Match: {0}".format(s))


def case4():
    print("Case #4: two texts from random letters.")
    s = 0
    for i in range(CNT_RANDOM_TEXTS):
        text1 = gen_random_letters(LEN_RANDOM_TEXT)
        text2 = gen_random_letters(LEN_RANDOM_TEXT)
        s += match_stat(text1, text2)
    s /= CNT_RANDOM_TEXTS
    print("Text length: {0}".format(LEN_RANDOM_TEXT))
    print("Match: {0}".format(s))


def case5():
    print("Case #5: two texts from random words.")
    s = 0
    for i in range(CNT_RANDOM_TEXTS):
        text1 = gen_random_words(LEN_RANDOM_TEXT)
        text2 = gen_random_words(LEN_RANDOM_TEXT)
        s += match_stat(text1, text2)
    s /= CNT_RANDOM_TEXTS
    print("Text length: {0}".format(LEN_RANDOM_TEXT))
    print("Match: {0}".format(s))

if __name__ == '__main__':
    case1()
    case2()
    case3()
    case4()
    case5()
\end{lstlisting}

\end{document}
